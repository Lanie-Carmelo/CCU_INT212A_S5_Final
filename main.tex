\documentclass[stu,12pt,floatsintext]{apa7}
% Language and citation setup
\usepackage[american]{babel}
\usepackage{csquotes}
\usepackage[style=apa,sortcites=true,sorting=nyt,backend=biber]{biblatex}
\DeclareLanguageMapping{american}{american-apa}
\addbibresource{references.bib}

% Font and encoding
\usepackage[T1]{fontenc}
\usepackage{newtxtext,newtxmath}  % Modern Times-like font with math support

% Document metadata
\title{Session 5 Final Assignment}
\author{Lanie Molinar}
\authorsaffiliations{Colorado Christian University}
\duedate{June 9, 2025,}
\course{Adult Studies Seminar II (INT-212A)}
\professor{Carrie Faust-Silva}

\begin{document}

\maketitle
\thispagestyle{plain}
\pagestyle{plain}

\section{Introduction}
Colorado Christian University's strategic priorities reflect a biblical, or Christian, worldview in many ways. Two priorities that stand out as particularly aligned with this worldview are: (1) honoring Christ and sharing His love around the world, and (2) teaching students to trust the Bible, live holy lives, and share the Gospel \parencite{coloradochristianuniversityStrategicPriorities}. These priorities influence every aspect of the university, from the curriculum, which integrates faith and biblical perspectives into every course, to the overall learning environment that encourages students to live out their faith in meaningful and practical ways. The university's commitment to a biblical worldview is evident in its mission to prepare students not just academically, but spiritually, equipping them to make a positive impact in the world. As a Christian who wants to deepen her faith and understanding of the Bible, I find these priorities particularly compelling. They resonate with my desire to grow in my relationship with Christ and to share His love with others. In this paper, I will explore how these priorities reflect a Christian worldview and reflect on how my experience in this course has shaped my understanding of what it means to attend CCU.
\section{Honoring Christ and Sharing His Love}
One of the most foundational teachings in the Bible is the commandment to love God with all our heart, soul, and mind \parencite[Matthew 22:37]{Tyndale1996}. This command, which includes honoring Christ as Lord and Savior, is central to the Christian faith and is clearly reflected in Colorado Christian University's emphasis on honoring Christ and sharing His love around the world. By placing Christ at the center of its mission, CCU aligns itself with the biblical call to live a Christ-centered life. This focus is not merely about academic achievement, but about cultivating a life that reflects Christ’s love and grace. It encourages students to view their education as a way to serve God and others, and to share the Gospel in their communities and beyond. This aligns with Jesus’ teaching that His followers are to be the salt of the earth and the light of the world \parencite[Matthew 5:13–16]{Tyndale1996}, demonstrating how faith can transform both individuals and society.
\section{Teaching Students to Trust the Bible}
The second strategic priority of CCU, teaching students to trust the Bible, live holy lives, and share the Gospel \parencite{coloradochristianuniversityStrategicPriorities}, is also deeply rooted in Scripture. The Bible itself emphasizes the importance of Scripture in guiding believers' lives. For example, 2 Timothy 3:16–17 states that all Scripture is God-breathed and useful for teaching, rebuking, correcting, and training in righteousness \parencite{Tyndale1996}. This passage underscores the belief that the Bible is not just a historical document but a living guide for faith and practice, meant to be returned to throughout our lives. By encouraging students to trust the Bible, CCU fosters a community where biblical truths are foundational to personal and academic growth. This commitment to Scripture equips students to navigate life's challenges with faith, wisdom, and integrity, reflecting the character of Christ in their actions and decisions. This emphasis on Scripture aligns with the evangelical principle of Biblicism, which holds that the Bible is the primary and authoritative source of spiritual truth and should be applied to every area of life and learning \parencite{FaustSilvaSession2}.
Since starting courses at CCU, I have been encouraged to engage with the Bible in a deeper way, learning not just to read it, but to apply its teachings to my life. This has helped me develop a more robust understanding of my faith and how it informs my actions and decisions. As I have studied the Bible more closely, I have seen my faith grow stronger, and I have grown bolder, less anxious, and more confident. This transformation in my spiritual life reflects the heart of CCU’s mission, to develop students who are grounded in biblical truth and equipped to live out their faith with courage and conviction.
\section{What It Means to Attend Colorado Christian University}
This is not my first attempt at going to college, but over the last few months, I have come to see how different CCU is from any other school I have attended. Faith truly is integrated into every course, and the professors are not only knowledgeable in their fields but also passionate about helping students grow in their faith. The learning environment is supportive and encouraging, fostering a sense of community among students who share a common goal of academic and spiritual growth. This has been a refreshing change from my previous experiences, where faith was often sidelined or treated as an afterthought. At CCU, I feel that my spiritual growth is just as important as my academic success, and this holistic approach to education has been transformative for me. I have learned to see my studies not just as a means to an end, but as a way to serve God and others, preparing me for a life of purpose and impact.
The course format and structure also work better for me than that of any other school I have tried. My multiple disabilities and health conditions can make taking more than one course at a time difficult, which often kept me from receiving financial aid at other schools, forcing me to drop out. At CCU, though, students take one course at a time, which allows me to focus on each subject without feeling overwhelmed. This format has made it possible for me to continue my education while managing my health, and I am grateful for the flexibility and understanding that CCU provides. The instructors and staff are also incredibly supportive, offering guidance and encouragement as I navigate my studies and my faith journey. They understand the challenges that students with disabilities face and are committed to helping us succeed both academically and spiritually.  
One of the most important things I discovered about CCU through this course is that attending this university means being part of a Christ-centered academic community where spiritual formation is just as important as intellectual development. This expanded my understanding of what it means to attend CCU, because I now see that education here is not just about earning a degree, but about growing in faith, character, and purpose. What may have the greatest impact on my approach to education at CCU is the realization that my studies are not separate from my faith, but an extension of it. This understanding motivates me to approach each assignment and discussion with a sense of purpose, knowing that I am growing not only as a student but also as a follower of Christ. As \textcite{Smith2024} points out, vocation is not merely about a career or occupation, but about responding to God’s unique call on our lives through faithful service. This realization will impact how I approach each course—not just as an academic requirement, but as an opportunity to grow emotionally and spiritually and serve others. It encourages me to be more intentional in applying biblical principles to my studies, to seek God’s guidance in my learning, and to view my education as a calling rather than just a task.
Since starting at CCU, I have been saying that I feel this is where I am supposed to be, and everything I have read and learned in this course has confirmed that feeling. I am excited to continue my journey at CCU, knowing that I am part of a community that values both academic excellence and spiritual growth. I look forward to growing in my relationship with Christ as I pursue my studies, and I am grateful for the opportunity to be part of a university that is committed to honoring Him in all that it does.
\section{Conclusion}
In conclusion, Colorado Christian University's strategic priorities reflect a biblical worldview that emphasizes honoring Christ and sharing His love, as well as teaching students to trust the Bible and live holy lives. These priorities are not just theoretical concepts but are lived out in the daily life of the university, shaping the academic environment and the spiritual formation of students. As I continue my journey at CCU, I am excited to see how these priorities will continue to influence my education and my faith. I am grateful for the opportunity to be part of a community that values both academic excellence and spiritual growth, and I look forward to growing in my relationship with Christ as I pursue my studies.
\printbibliography

\end{document}